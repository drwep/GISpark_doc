% Generated by Sphinx.
\def\sphinxdocclass{report}
\documentclass[letterpaper,10pt,english]{sphinxmanual}

\usepackage[utf8]{inputenc}
\ifdefined\DeclareUnicodeCharacter
  \DeclareUnicodeCharacter{00A0}{\nobreakspace}
\else\fi
\usepackage{cmap}
\usepackage[T1]{fontenc}
\usepackage{amsmath,amssymb}
\usepackage{babel}
\usepackage{times}
\usepackage[Sonny]{fncychap}
\usepackage{longtable}
\usepackage{sphinx}
\usepackage{multirow}
\usepackage{eqparbox}


\addto\captionsenglish{\renewcommand{\figurename}{图 }}
\addto\captionsenglish{\renewcommand{\tablename}{表 }}
\SetupFloatingEnvironment{literal-block}{name=列表 }

\addto\extrasenglish{\def\pageautorefname{page}}

\setcounter{tocdepth}{1}

% Jupyter Notebook prompt colors
\definecolor{nbsphinxin}{HTML}{303F9F}
\definecolor{nbsphinxout}{HTML}{D84315}
% ANSI colors for output streams and traceback highlighting
\definecolor{ansi-black}{HTML}{3E424D}
\definecolor{ansi-black-intense}{HTML}{282C36}
\definecolor{ansi-red}{HTML}{E75C58}
\definecolor{ansi-red-intense}{HTML}{B22B31}
\definecolor{ansi-green}{HTML}{00A250}
\definecolor{ansi-green-intense}{HTML}{007427}
\definecolor{ansi-yellow}{HTML}{DDB62B}
\definecolor{ansi-yellow-intense}{HTML}{B27D12}
\definecolor{ansi-blue}{HTML}{208FFB}
\definecolor{ansi-blue-intense}{HTML}{0065CA}
\definecolor{ansi-magenta}{HTML}{D160C4}
\definecolor{ansi-magenta-intense}{HTML}{A03196}
\definecolor{ansi-cyan}{HTML}{60C6C8}
\definecolor{ansi-cyan-intense}{HTML}{258F8F}
\definecolor{ansi-white}{HTML}{C5C1B4}
\definecolor{ansi-white-intense}{HTML}{A1A6B2}


\hypersetup{unicode=true}
#\usepackage{CJKutf8}
\AtBeginDocument{\begin{CJK}{UTF8}{gbsn}}
\AtEndDocument{\end{CJK}}


\title{GISpark Documentation}
\date{4月 27, 2016}
\release{0.1}
\author{openthings}
\newcommand{\sphinxlogo}{}
\renewcommand{\releasename}{发布}
\makeindex

\makeatletter
\def\PYG@reset{\let\PYG@it=\relax \let\PYG@bf=\relax%
    \let\PYG@ul=\relax \let\PYG@tc=\relax%
    \let\PYG@bc=\relax \let\PYG@ff=\relax}
\def\PYG@tok#1{\csname PYG@tok@#1\endcsname}
\def\PYG@toks#1+{\ifx\relax#1\empty\else%
    \PYG@tok{#1}\expandafter\PYG@toks\fi}
\def\PYG@do#1{\PYG@bc{\PYG@tc{\PYG@ul{%
    \PYG@it{\PYG@bf{\PYG@ff{#1}}}}}}}
\def\PYG#1#2{\PYG@reset\PYG@toks#1+\relax+\PYG@do{#2}}

\expandafter\def\csname PYG@tok@w\endcsname{\def\PYG@tc##1{\textcolor[rgb]{0.73,0.73,0.73}{##1}}}
\expandafter\def\csname PYG@tok@ss\endcsname{\def\PYG@tc##1{\textcolor[rgb]{0.32,0.47,0.09}{##1}}}
\expandafter\def\csname PYG@tok@sd\endcsname{\let\PYG@it=\textit\def\PYG@tc##1{\textcolor[rgb]{0.25,0.44,0.63}{##1}}}
\expandafter\def\csname PYG@tok@nt\endcsname{\let\PYG@bf=\textbf\def\PYG@tc##1{\textcolor[rgb]{0.02,0.16,0.45}{##1}}}
\expandafter\def\csname PYG@tok@mo\endcsname{\def\PYG@tc##1{\textcolor[rgb]{0.13,0.50,0.31}{##1}}}
\expandafter\def\csname PYG@tok@cp\endcsname{\def\PYG@tc##1{\textcolor[rgb]{0.00,0.44,0.13}{##1}}}
\expandafter\def\csname PYG@tok@no\endcsname{\def\PYG@tc##1{\textcolor[rgb]{0.38,0.68,0.84}{##1}}}
\expandafter\def\csname PYG@tok@kp\endcsname{\def\PYG@tc##1{\textcolor[rgb]{0.00,0.44,0.13}{##1}}}
\expandafter\def\csname PYG@tok@bp\endcsname{\def\PYG@tc##1{\textcolor[rgb]{0.00,0.44,0.13}{##1}}}
\expandafter\def\csname PYG@tok@il\endcsname{\def\PYG@tc##1{\textcolor[rgb]{0.13,0.50,0.31}{##1}}}
\expandafter\def\csname PYG@tok@gu\endcsname{\let\PYG@bf=\textbf\def\PYG@tc##1{\textcolor[rgb]{0.50,0.00,0.50}{##1}}}
\expandafter\def\csname PYG@tok@k\endcsname{\let\PYG@bf=\textbf\def\PYG@tc##1{\textcolor[rgb]{0.00,0.44,0.13}{##1}}}
\expandafter\def\csname PYG@tok@nc\endcsname{\let\PYG@bf=\textbf\def\PYG@tc##1{\textcolor[rgb]{0.05,0.52,0.71}{##1}}}
\expandafter\def\csname PYG@tok@m\endcsname{\def\PYG@tc##1{\textcolor[rgb]{0.13,0.50,0.31}{##1}}}
\expandafter\def\csname PYG@tok@c\endcsname{\let\PYG@it=\textit\def\PYG@tc##1{\textcolor[rgb]{0.25,0.50,0.56}{##1}}}
\expandafter\def\csname PYG@tok@gt\endcsname{\def\PYG@tc##1{\textcolor[rgb]{0.00,0.27,0.87}{##1}}}
\expandafter\def\csname PYG@tok@s1\endcsname{\def\PYG@tc##1{\textcolor[rgb]{0.25,0.44,0.63}{##1}}}
\expandafter\def\csname PYG@tok@nd\endcsname{\let\PYG@bf=\textbf\def\PYG@tc##1{\textcolor[rgb]{0.33,0.33,0.33}{##1}}}
\expandafter\def\csname PYG@tok@ch\endcsname{\let\PYG@it=\textit\def\PYG@tc##1{\textcolor[rgb]{0.25,0.50,0.56}{##1}}}
\expandafter\def\csname PYG@tok@cpf\endcsname{\let\PYG@it=\textit\def\PYG@tc##1{\textcolor[rgb]{0.25,0.50,0.56}{##1}}}
\expandafter\def\csname PYG@tok@gp\endcsname{\let\PYG@bf=\textbf\def\PYG@tc##1{\textcolor[rgb]{0.78,0.36,0.04}{##1}}}
\expandafter\def\csname PYG@tok@nv\endcsname{\def\PYG@tc##1{\textcolor[rgb]{0.73,0.38,0.84}{##1}}}
\expandafter\def\csname PYG@tok@err\endcsname{\def\PYG@bc##1{\setlength{\fboxsep}{0pt}\fcolorbox[rgb]{1.00,0.00,0.00}{1,1,1}{\strut ##1}}}
\expandafter\def\csname PYG@tok@kc\endcsname{\let\PYG@bf=\textbf\def\PYG@tc##1{\textcolor[rgb]{0.00,0.44,0.13}{##1}}}
\expandafter\def\csname PYG@tok@ge\endcsname{\let\PYG@it=\textit}
\expandafter\def\csname PYG@tok@se\endcsname{\let\PYG@bf=\textbf\def\PYG@tc##1{\textcolor[rgb]{0.25,0.44,0.63}{##1}}}
\expandafter\def\csname PYG@tok@nf\endcsname{\def\PYG@tc##1{\textcolor[rgb]{0.02,0.16,0.49}{##1}}}
\expandafter\def\csname PYG@tok@sb\endcsname{\def\PYG@tc##1{\textcolor[rgb]{0.25,0.44,0.63}{##1}}}
\expandafter\def\csname PYG@tok@vc\endcsname{\def\PYG@tc##1{\textcolor[rgb]{0.73,0.38,0.84}{##1}}}
\expandafter\def\csname PYG@tok@gd\endcsname{\def\PYG@tc##1{\textcolor[rgb]{0.63,0.00,0.00}{##1}}}
\expandafter\def\csname PYG@tok@gh\endcsname{\let\PYG@bf=\textbf\def\PYG@tc##1{\textcolor[rgb]{0.00,0.00,0.50}{##1}}}
\expandafter\def\csname PYG@tok@sh\endcsname{\def\PYG@tc##1{\textcolor[rgb]{0.25,0.44,0.63}{##1}}}
\expandafter\def\csname PYG@tok@ne\endcsname{\def\PYG@tc##1{\textcolor[rgb]{0.00,0.44,0.13}{##1}}}
\expandafter\def\csname PYG@tok@sc\endcsname{\def\PYG@tc##1{\textcolor[rgb]{0.25,0.44,0.63}{##1}}}
\expandafter\def\csname PYG@tok@s2\endcsname{\def\PYG@tc##1{\textcolor[rgb]{0.25,0.44,0.63}{##1}}}
\expandafter\def\csname PYG@tok@vi\endcsname{\def\PYG@tc##1{\textcolor[rgb]{0.73,0.38,0.84}{##1}}}
\expandafter\def\csname PYG@tok@mi\endcsname{\def\PYG@tc##1{\textcolor[rgb]{0.13,0.50,0.31}{##1}}}
\expandafter\def\csname PYG@tok@nl\endcsname{\let\PYG@bf=\textbf\def\PYG@tc##1{\textcolor[rgb]{0.00,0.13,0.44}{##1}}}
\expandafter\def\csname PYG@tok@kr\endcsname{\let\PYG@bf=\textbf\def\PYG@tc##1{\textcolor[rgb]{0.00,0.44,0.13}{##1}}}
\expandafter\def\csname PYG@tok@vg\endcsname{\def\PYG@tc##1{\textcolor[rgb]{0.73,0.38,0.84}{##1}}}
\expandafter\def\csname PYG@tok@o\endcsname{\def\PYG@tc##1{\textcolor[rgb]{0.40,0.40,0.40}{##1}}}
\expandafter\def\csname PYG@tok@nn\endcsname{\let\PYG@bf=\textbf\def\PYG@tc##1{\textcolor[rgb]{0.05,0.52,0.71}{##1}}}
\expandafter\def\csname PYG@tok@kt\endcsname{\def\PYG@tc##1{\textcolor[rgb]{0.56,0.13,0.00}{##1}}}
\expandafter\def\csname PYG@tok@nb\endcsname{\def\PYG@tc##1{\textcolor[rgb]{0.00,0.44,0.13}{##1}}}
\expandafter\def\csname PYG@tok@gi\endcsname{\def\PYG@tc##1{\textcolor[rgb]{0.00,0.63,0.00}{##1}}}
\expandafter\def\csname PYG@tok@mb\endcsname{\def\PYG@tc##1{\textcolor[rgb]{0.13,0.50,0.31}{##1}}}
\expandafter\def\csname PYG@tok@go\endcsname{\def\PYG@tc##1{\textcolor[rgb]{0.20,0.20,0.20}{##1}}}
\expandafter\def\csname PYG@tok@sr\endcsname{\def\PYG@tc##1{\textcolor[rgb]{0.14,0.33,0.53}{##1}}}
\expandafter\def\csname PYG@tok@na\endcsname{\def\PYG@tc##1{\textcolor[rgb]{0.25,0.44,0.63}{##1}}}
\expandafter\def\csname PYG@tok@mh\endcsname{\def\PYG@tc##1{\textcolor[rgb]{0.13,0.50,0.31}{##1}}}
\expandafter\def\csname PYG@tok@si\endcsname{\let\PYG@it=\textit\def\PYG@tc##1{\textcolor[rgb]{0.44,0.63,0.82}{##1}}}
\expandafter\def\csname PYG@tok@s\endcsname{\def\PYG@tc##1{\textcolor[rgb]{0.25,0.44,0.63}{##1}}}
\expandafter\def\csname PYG@tok@gs\endcsname{\let\PYG@bf=\textbf}
\expandafter\def\csname PYG@tok@cm\endcsname{\let\PYG@it=\textit\def\PYG@tc##1{\textcolor[rgb]{0.25,0.50,0.56}{##1}}}
\expandafter\def\csname PYG@tok@gr\endcsname{\def\PYG@tc##1{\textcolor[rgb]{1.00,0.00,0.00}{##1}}}
\expandafter\def\csname PYG@tok@c1\endcsname{\let\PYG@it=\textit\def\PYG@tc##1{\textcolor[rgb]{0.25,0.50,0.56}{##1}}}
\expandafter\def\csname PYG@tok@mf\endcsname{\def\PYG@tc##1{\textcolor[rgb]{0.13,0.50,0.31}{##1}}}
\expandafter\def\csname PYG@tok@kd\endcsname{\let\PYG@bf=\textbf\def\PYG@tc##1{\textcolor[rgb]{0.00,0.44,0.13}{##1}}}
\expandafter\def\csname PYG@tok@ni\endcsname{\let\PYG@bf=\textbf\def\PYG@tc##1{\textcolor[rgb]{0.84,0.33,0.22}{##1}}}
\expandafter\def\csname PYG@tok@cs\endcsname{\def\PYG@tc##1{\textcolor[rgb]{0.25,0.50,0.56}{##1}}\def\PYG@bc##1{\setlength{\fboxsep}{0pt}\colorbox[rgb]{1.00,0.94,0.94}{\strut ##1}}}
\expandafter\def\csname PYG@tok@sx\endcsname{\def\PYG@tc##1{\textcolor[rgb]{0.78,0.36,0.04}{##1}}}
\expandafter\def\csname PYG@tok@kn\endcsname{\let\PYG@bf=\textbf\def\PYG@tc##1{\textcolor[rgb]{0.00,0.44,0.13}{##1}}}
\expandafter\def\csname PYG@tok@ow\endcsname{\let\PYG@bf=\textbf\def\PYG@tc##1{\textcolor[rgb]{0.00,0.44,0.13}{##1}}}

\def\PYGZbs{\char`\\}
\def\PYGZus{\char`\_}
\def\PYGZob{\char`\{}
\def\PYGZcb{\char`\}}
\def\PYGZca{\char`\^}
\def\PYGZam{\char`\&}
\def\PYGZlt{\char`\<}
\def\PYGZgt{\char`\>}
\def\PYGZsh{\char`\#}
\def\PYGZpc{\char`\%}
\def\PYGZdl{\char`\$}
\def\PYGZhy{\char`\-}
\def\PYGZsq{\char`\'}
\def\PYGZdq{\char`\"}
\def\PYGZti{\char`\~}
% for compatibility with earlier versions
\def\PYGZat{@}
\def\PYGZlb{[}
\def\PYGZrb{]}
\makeatother

\renewcommand\PYGZsq{\textquotesingle}

\begin{document}

\maketitle
\tableofcontents
\phantomsection\label{index::doc}


Contents:


\chapter{1、运行环境}
\label{index:gispark}\label{index:id1}

\section{1.1 Docker和Mesos}
\label{index:dockermesos}

\section{1.2 Python和Jupyter}
\label{index:pythonjupyter}
通过功能强大的Notebook进行远程数据分析。
\begin{itemize}
\item {} 
Notebook

\item {} 
ReadTheDocs

\end{itemize}


\section{1.3 Spark分布式计算环境}
\label{index:spark}

\chapter{2、开放数据}
\label{index:id2}

\section{Hello World}
\label{helloworld:hello-world}\label{helloworld::doc}

\section{Markdown test}
\label{markit:markdown-test}\label{markit::doc}\begin{itemize}
\item {} 
Markdown test

\end{itemize}


\section{《Data Master》}
\label{pystart_databasic:_u300aData-Master_u300b}\label{pystart_databasic::doc}\begin{quote}
\end{quote}


\subsection{基础篇—Python快速入门}
\label{pystart_databasic:_u57fa_u7840_u7bc7_u2014Python_u5feb_u901f_u5165_u95e8}\begin{quote}
\end{quote}


\subsubsection{List,使用{[}a,b,c,...{]}方式声明。}
\label{pystart_databasic:List_uff0c_u4f7f_u7528_a,b,c,...__u65b9_u5f0f_u58f0_u660e_u3002}
列表是基础的数据结构。

\begin{OriginalVerbatim}[commandchars=\\\{\}]
\textcolor{nbsphinxin}{In [9]: }alist = [0,1,2,3,4]
        print(\PYGZdq{}总计:\PYGZdq{},len(alist))
        print(\PYGZdq{}元素:\PYGZdq{}, alist)
\end{OriginalVerbatim}
% This comment is needed to force a line break for adjacent ANSI cells
\begin{OriginalVerbatim}[commandchars=\\\{\}]
总计: 5
元素: [0, 1, 2, 3, 4]
\end{OriginalVerbatim}
字符串的列表。

\begin{OriginalVerbatim}[commandchars=\\\{\}]
\textcolor{nbsphinxin}{In [20]: }colours = [\PYGZdq{}red\PYGZdq{},\PYGZdq{}green\PYGZdq{},\PYGZdq{}blue\PYGZdq{}]
         for colour in colours:
             print(colour)
\end{OriginalVerbatim}
% This comment is needed to force a line break for adjacent ANSI cells
\begin{OriginalVerbatim}[commandchars=\\\{\}]
red
green
blue
\end{OriginalVerbatim}
列表的传统方式遍历。

\begin{OriginalVerbatim}[commandchars=\\\{\}]
\textcolor{nbsphinxin}{In [24]: }for i in range(0,len(alist)):
             print(alist[i])
\end{OriginalVerbatim}
% This comment is needed to force a line break for adjacent ANSI cells
\begin{OriginalVerbatim}[commandchars=\\\{\}]
0
1
2
3
4
\end{OriginalVerbatim}
列表的递归方式遍历。

\begin{OriginalVerbatim}[commandchars=\\\{\}]
\textcolor{nbsphinxin}{In [25]: }for i in alist:
             print(i)
\end{OriginalVerbatim}
% This comment is needed to force a line break for adjacent ANSI cells
\begin{OriginalVerbatim}[commandchars=\\\{\}]
0
1
2
3
4
\end{OriginalVerbatim}
可以直接调用一个列表。

\begin{OriginalVerbatim}[commandchars=\\\{\}]
\textcolor{nbsphinxin}{In [26]: }for obj in [0,1,2,3,4]:
             print(obj)
\end{OriginalVerbatim}
% This comment is needed to force a line break for adjacent ANSI cells
\begin{OriginalVerbatim}[commandchars=\\\{\}]
0
1
2
3
4
\end{OriginalVerbatim}

\paragraph{创建一个矩阵。}
\label{pystart_databasic:_u521b_u5efa_u4e00_u4e2a_u77e9_u9635_u3002}
\begin{OriginalVerbatim}[commandchars=\\\{\}]
\textcolor{nbsphinxin}{In [63]: }olist = [[11,12,13],[21,22,23],[31,32,33]]
         for row in olist:
             print(row)
\end{OriginalVerbatim}
% This comment is needed to force a line break for adjacent ANSI cells
\begin{OriginalVerbatim}[commandchars=\\\{\}]
[11, 12, 13]
[21, 22, 23]
[31, 32, 33]
\end{OriginalVerbatim}

\paragraph{生成一个数据序列。在做数值检验时很有用。}
\label{pystart_databasic:_u751f_u6210_u4e00_u4e2a_u6570_u636e_u5e8f_u5217_u3002_u5728_u505a_u6570_u503c_u68c0_u9a8c_u65f6_u5f88_u6709_u7528_u3002}
\begin{OriginalVerbatim}[commandchars=\\\{\}]
\textcolor{nbsphinxin}{In [43]: }for obj in range(5):
             print(obj)
\end{OriginalVerbatim}
% This comment is needed to force a line break for adjacent ANSI cells
\begin{OriginalVerbatim}[commandchars=\\\{\}]
0
1
2
3
4
\end{OriginalVerbatim}

\paragraph{生成一个数据序列:range(开始值,结束值,步长)}
\label{pystart_databasic:_u751f_u6210_u4e00_u4e2a_u6570_u636e_u5e8f_u5217_uff1arange(_u5f00_u59cb_u503c_uff0c_u7ed3_u675f_u503c_uff0c_u6b65_u957f)}
\begin{OriginalVerbatim}[commandchars=\\\{\}]
\textcolor{nbsphinxin}{In [27]: }for obj in range(5,10,2):
             print(obj)
\end{OriginalVerbatim}
% This comment is needed to force a line break for adjacent ANSI cells
\begin{OriginalVerbatim}[commandchars=\\\{\}]
5
7
9
\end{OriginalVerbatim}

\subparagraph{String as a list of char. 字符串是字符数组。}
\label{pystart_databasic:String-as-a-list-of-char.-_u5b57_u7b26_u4e32_u662f_u5b57_u7b26_u6570_u7ec4_u3002}
\begin{OriginalVerbatim}[commandchars=\\\{\}]
\textcolor{nbsphinxin}{In [7]: }name=\PYGZsq{}BeginMan\PYGZsq{}
        for obj in range(len(name)):
            print(\PYGZsq{}(\PYGZpc{}d)\PYGZsq{} \PYGZpc{}obj,name[obj])
\end{OriginalVerbatim}
% This comment is needed to force a line break for adjacent ANSI cells
\begin{OriginalVerbatim}[commandchars=\\\{\}]
(0) B
(1) e
(2) g
(3) i
(4) n
(5) M
(6) a
(7) n
\end{OriginalVerbatim}

\subsection{Dictionary,词典,{key:value,...}}
\label{pystart_databasic:Dictionary_uff0c_u8bcd_u5178_uff0c_uff5bkey:value,..._uff5d}
词典数据就是一系列k:v值对的集合。

\begin{OriginalVerbatim}[commandchars=\\\{\}]
\textcolor{nbsphinxin}{In [30]: }dict = \PYGZob{}\PYGZsq{}name\PYGZsq{}:\PYGZsq{}BeginMan\PYGZsq{},\PYGZsq{}job\PYGZsq{}:\PYGZsq{}pythoner\PYGZsq{},\PYGZsq{}age\PYGZsq{}:22\PYGZcb{}
         
         print(\PYGZdq{}Dict Length: \PYGZdq{},len(dict))
         print(dict)
\end{OriginalVerbatim}
% This comment is needed to force a line break for adjacent ANSI cells
\begin{OriginalVerbatim}[commandchars=\\\{\}]
Dict Length:  3
{'age': 22, 'job': 'pythoner', 'name': 'BeginMan'}
\end{OriginalVerbatim}
\textbf{*注意:}上面的词典数据的输出与json表示是完全一致的,后面在json会专门介绍。*

词典的遍历:

\begin{OriginalVerbatim}[commandchars=\\\{\}]
\textcolor{nbsphinxin}{In [71]: }dict[\PYGZdq{}name\PYGZdq{}]
\end{OriginalVerbatim}

\begin{OriginalVerbatim}[commandchars=\\\{\}]
\textcolor{nbsphinxout}{Out[71]: }\PYGZsq{}BeginMan\PYGZsq{}
\end{OriginalVerbatim}

\begin{OriginalVerbatim}[commandchars=\\\{\}]
\textcolor{nbsphinxin}{In [41]: }print(\PYGZdq{}Key\PYGZdq{},\PYGZdq{}\PYGZbs{}t Value\PYGZdq{})
         print(\PYGZdq{}=================\PYGZdq{})
         for key in dict:
             print(key,\PYGZdq{}\PYGZbs{}t\PYGZdq{},dict[i])
\end{OriginalVerbatim}
% This comment is needed to force a line break for adjacent ANSI cells
\begin{OriginalVerbatim}[commandchars=\\\{\}]
Key      Value
=================
age      BeginMan
job      BeginMan
name     BeginMan
\end{OriginalVerbatim}
dict的每一个item(obj)是一个二元组(下面介绍元组)。

\begin{OriginalVerbatim}[commandchars=\\\{\}]
\textcolor{nbsphinxin}{In [31]: }for obj in dict.items():
             print(obj)
\end{OriginalVerbatim}
% This comment is needed to force a line break for adjacent ANSI cells
\begin{OriginalVerbatim}[commandchars=\\\{\}]
('age', 22)
('job', 'pythoner')
('name', 'BeginMan')
\end{OriginalVerbatim}
\begin{OriginalVerbatim}[commandchars=\\\{\}]
\textcolor{nbsphinxin}{In [46]: }for k,v in dict.items():
             print(k,v)
\end{OriginalVerbatim}
% This comment is needed to force a line break for adjacent ANSI cells
\begin{OriginalVerbatim}[commandchars=\\\{\}]
age 22
job pythoner
name BeginMan
\end{OriginalVerbatim}
\begin{OriginalVerbatim}[commandchars=\\\{\}]
\textcolor{nbsphinxin}{In [72]: }import json
         
         j = json.dumps(dict)
         print(repr(j))
\end{OriginalVerbatim}
% This comment is needed to force a line break for adjacent ANSI cells
\begin{OriginalVerbatim}[commandchars=\\\{\}]
'{"age": 22, "job": "pythoner", "name": "BeginMan"}'
\end{OriginalVerbatim}
\begin{OriginalVerbatim}[commandchars=\\\{\}]
\textcolor{nbsphinxin}{In [53]: }d = json.loads(\PYGZsq{}\PYGZob{}\PYGZdq{}age\PYGZdq{}: 22, \PYGZdq{}job\PYGZdq{}: \PYGZdq{}pythoner\PYGZdq{}, \PYGZdq{}name\PYGZdq{}: \PYGZdq{}BeginMan\PYGZdq{}\PYGZcb{}\PYGZsq{})
         print(\PYGZdq{}Type of d: \PYGZdq{}, type(d))
         print(d)
\end{OriginalVerbatim}
% This comment is needed to force a line break for adjacent ANSI cells
\begin{OriginalVerbatim}[commandchars=\\\{\}]
Type of d:  <class 'dict'>
{'name': 'BeginMan', 'job': 'pythoner', 'age': 22}
\end{OriginalVerbatim}

\subsection{tuple,(obj1,obj2,...),元组}
\label{pystart_databasic:tuple_uff0c(obj1,obj2,...)_uff0c_u5143_u7ec4}
一个元组可包含多种类型的对象,不可修改。

\begin{OriginalVerbatim}[commandchars=\\\{\}]
\textcolor{nbsphinxin}{In [59]: }tup = \PYGZsq{}abc\PYGZsq{},1,2,\PYGZsq{}x\PYGZsq{},True
\end{OriginalVerbatim}

\begin{OriginalVerbatim}[commandchars=\\\{\}]
\textcolor{nbsphinxin}{In [60]: }len(tup),tup
\end{OriginalVerbatim}

\begin{OriginalVerbatim}[commandchars=\\\{\}]
\textcolor{nbsphinxout}{Out[60]: }(5, (\PYGZsq{}abc\PYGZsq{}, 1, 2, \PYGZsq{}x\PYGZsq{}, True))
\end{OriginalVerbatim}

\begin{OriginalVerbatim}[commandchars=\\\{\}]
\textcolor{nbsphinxin}{In [56]: }x,y =1,2
\end{OriginalVerbatim}

\begin{OriginalVerbatim}[commandchars=\\\{\}]
\textcolor{nbsphinxin}{In [10]: }x,y
\end{OriginalVerbatim}

\begin{OriginalVerbatim}[commandchars=\\\{\}]
\textcolor{nbsphinxout}{Out[10]: }(1, 2)
\end{OriginalVerbatim}


\subsubsection{一个字典、元组构成的复合列表对象。}
\label{pystart_databasic:_u4e00_u4e2a_u5b57_u5178_u3001_u5143_u7ec4_u6784_u6210_u7684_u590d_u5408_u5217_u8868_u5bf9_u8c61_u3002}
\begin{OriginalVerbatim}[commandchars=\\\{\}]
\textcolor{nbsphinxin}{In [67]: }ao = [\PYGZob{}\PYGZdq{}k1\PYGZdq{}:\PYGZdq{}key\PYGZdq{},\PYGZdq{}k2\PYGZdq{}:2\PYGZcb{},(3,\PYGZdq{}element\PYGZdq{})]
         ao
\end{OriginalVerbatim}

\begin{OriginalVerbatim}[commandchars=\\\{\}]
\textcolor{nbsphinxout}{Out[67]: }[\PYGZob{}\PYGZsq{}k1\PYGZsq{}: \PYGZsq{}key\PYGZsq{}, \PYGZsq{}k2\PYGZsq{}: 2\PYGZcb{}, (3, \PYGZsq{}element\PYGZsq{})]
\end{OriginalVerbatim}


\paragraph{访问这个符合对象的值。}
\label{pystart_databasic:_u8bbf_u95ee_u8fd9_u4e2a_u7b26_u5408_u5bf9_u8c61_u7684_u503c_u3002}len(ao),ao[0]["k1"],ao[1][0],ao[1][1]
从上面可以看出,python的数据结构是非常灵活的,是数据探索和分析、处理的利器。


\section{Pandas\_QuickStart}
\label{pandas_quickstart:Pandas_QuickStart}\label{pandas_quickstart::doc}
\begin{DUlineblock}{0em}
\item[] Origin from \url{http://pandas.pydata.org/pandas-docs/stable/}
\item[] by
\href{http://my.oschina.net/u/2306127/blog?catalog=3420733}{openthings@163.com},
2016-04.
\end{DUlineblock}


\subsection{6.1 Object Creation}
\label{pandas_quickstart:6.1-Object-Creation}
Creating a Series by passing a list of values, letting pandas create a
default integer index:

\begin{OriginalVerbatim}[commandchars=\\\{\}]
\textcolor{nbsphinxin}{In [1]: }import pandas as pd
        import numpy as np
        import matplotlib.pyplot as plt
        
        s = pd.Series([1,3,5,np.nan,6,8])
        s
\end{OriginalVerbatim}

\begin{OriginalVerbatim}[commandchars=\\\{\}]
\textcolor{nbsphinxout}{Out[1]: }0    1.0
        1    3.0
        2    5.0
        3    NaN
        4    6.0
        5    8.0
        dtype: float64
\end{OriginalVerbatim}

Creating a DataFrame by passing a numpy array, with a datetime index and
labeled columns:

\begin{OriginalVerbatim}[commandchars=\\\{\}]
\textcolor{nbsphinxin}{In [2]: }dates = pd.date\PYGZus{}range(\PYGZsq{}20130101\PYGZsq{}, periods=6)
        dates
\end{OriginalVerbatim}

\begin{OriginalVerbatim}[commandchars=\\\{\}]
\textcolor{nbsphinxout}{Out[2]: }DatetimeIndex([\PYGZsq{}2013\PYGZhy{}01\PYGZhy{}01\PYGZsq{}, \PYGZsq{}2013\PYGZhy{}01\PYGZhy{}02\PYGZsq{}, \PYGZsq{}2013\PYGZhy{}01\PYGZhy{}03\PYGZsq{}, \PYGZsq{}2013\PYGZhy{}01\PYGZhy{}04\PYGZsq{},
                       \PYGZsq{}2013\PYGZhy{}01\PYGZhy{}05\PYGZsq{}, \PYGZsq{}2013\PYGZhy{}01\PYGZhy{}06\PYGZsq{}],
                      dtype=\PYGZsq{}datetime64[ns]\PYGZsq{}, freq=\PYGZsq{}D\PYGZsq{})
\end{OriginalVerbatim}

\begin{OriginalVerbatim}[commandchars=\\\{\}]
\textcolor{nbsphinxin}{In [7]: }df = pd.DataFrame(np.random.randn(6,4), index=dates, columns=list(\PYGZsq{}ABCD\PYGZsq{}))
        
        df
\end{OriginalVerbatim}

Creating a DataFrame by passing a dict of objects that can be converted
to series-like.

\begin{OriginalVerbatim}[commandchars=\\\{\}]
\textcolor{nbsphinxin}{In [8]: }df2 = pd.DataFrame(\PYGZob{} \PYGZsq{}A\PYGZsq{} : 1.,
        \PYGZsq{}B\PYGZsq{} : pd.Timestamp(\PYGZsq{}20130102\PYGZsq{}),
        \PYGZsq{}C\PYGZsq{} : pd.Series(1,index=list(range(4)),dtype=\PYGZsq{}float32\PYGZsq{}),
        \PYGZsq{}D\PYGZsq{} : np.array([3] * 4,dtype=\PYGZsq{}int32\PYGZsq{}),
        \PYGZsq{}E\PYGZsq{} : pd.Categorical([\PYGZdq{}test\PYGZdq{},\PYGZdq{}train\PYGZdq{},\PYGZdq{}test\PYGZdq{},\PYGZdq{}train\PYGZdq{}]),
        \PYGZsq{}F\PYGZsq{} : \PYGZsq{}foo\PYGZsq{} \PYGZcb{})
        
        df2
\end{OriginalVerbatim}

\begin{OriginalVerbatim}[commandchars=\\\{\}]
\textcolor{nbsphinxin}{In [9]: }df2.dtypes
\end{OriginalVerbatim}

\begin{OriginalVerbatim}[commandchars=\\\{\}]
\textcolor{nbsphinxout}{Out[9]: }A           float64
        B    datetime64[ns]
        C           float32
        D             int32
        E          category
        F            object
        dtype: object
\end{OriginalVerbatim}

If you’re using IPython, tab completion for column names (as well as
public attributes) is automatically enabled. Here’s a subset of the
attributes that will be completed:

\begin{Verbatim}[commandchars=\\\{\}]
\PYG{n}{In} \PYG{p}{[}\PYG{l+m+mi}{13}\PYG{p}{]}\PYG{p}{:} \PYG{n}{df2}\PYG{o}{.}\PYG{o}{\PYGZlt{}}\PYG{n}{TAB}\PYG{o}{\PYGZgt{}}
\end{Verbatim}

\begin{OriginalVerbatim}[commandchars=\\\{\}]
\textcolor{nbsphinxin}{In [11]: }df2.
\end{OriginalVerbatim}

\begin{OriginalVerbatim}[commandchars=\\\{\}]
\textcolor{nbsphinxout}{Out[11]: }0    1.0
         1    1.0
         2    1.0
         3    1.0
         Name: A, dtype: float64
\end{OriginalVerbatim}

As you can see, the columns A, B, C, and D are automatically tab
completed. E is there as well; the rest of the attributes have been
truncated for brevity.


\subsection{6.2 Viewing Data}
\label{pandas_quickstart:6.2-Viewing-Data}
\begin{OriginalVerbatim}[commandchars=\\\{\}]
\textcolor{nbsphinxin}{In [14]: }df.head()
\end{OriginalVerbatim}

\begin{OriginalVerbatim}[commandchars=\\\{\}]
\textcolor{nbsphinxin}{In [15]: }df.tail(3)
\end{OriginalVerbatim}

\begin{OriginalVerbatim}[commandchars=\\\{\}]
\textcolor{nbsphinxin}{In [16]: }df.index
\end{OriginalVerbatim}

\begin{OriginalVerbatim}[commandchars=\\\{\}]
\textcolor{nbsphinxout}{Out[16]: }DatetimeIndex([\PYGZsq{}2013\PYGZhy{}01\PYGZhy{}01\PYGZsq{}, \PYGZsq{}2013\PYGZhy{}01\PYGZhy{}02\PYGZsq{}, \PYGZsq{}2013\PYGZhy{}01\PYGZhy{}03\PYGZsq{}, \PYGZsq{}2013\PYGZhy{}01\PYGZhy{}04\PYGZsq{},
                        \PYGZsq{}2013\PYGZhy{}01\PYGZhy{}05\PYGZsq{}, \PYGZsq{}2013\PYGZhy{}01\PYGZhy{}06\PYGZsq{}],
                       dtype=\PYGZsq{}datetime64[ns]\PYGZsq{}, freq=\PYGZsq{}D\PYGZsq{})
\end{OriginalVerbatim}

\begin{OriginalVerbatim}[commandchars=\\\{\}]
\textcolor{nbsphinxin}{In [17]: }df.values
\end{OriginalVerbatim}

\begin{OriginalVerbatim}[commandchars=\\\{\}]
\textcolor{nbsphinxout}{Out[17]: }array([[\PYGZhy{}1.33401275, \PYGZhy{}0.34829657,  0.38865407, \PYGZhy{}0.22596701],
                [\PYGZhy{}0.13997444, \PYGZhy{}1.34778853,  0.81707707,  0.19247685],
                [\PYGZhy{}1.0827386 , \PYGZhy{}0.5441047 , \PYGZhy{}1.42388302, \PYGZhy{}1.24736743],
                [ 0.03478847, \PYGZhy{}0.67722051,  0.12044917,  0.7943414 ],
                [ 0.42854678, \PYGZhy{}0.61015602, \PYGZhy{}0.95089113, \PYGZhy{}0.0580473 ],
                [ 0.12563068, \PYGZhy{}0.11665286, \PYGZhy{}0.54457518, \PYGZhy{}1.57878468]])
\end{OriginalVerbatim}

\begin{OriginalVerbatim}[commandchars=\\\{\}]
\textcolor{nbsphinxin}{In [18]: }df.describe()
\end{OriginalVerbatim}

\begin{OriginalVerbatim}[commandchars=\\\{\}]
\textcolor{nbsphinxin}{In [19]: }df.T
\end{OriginalVerbatim}

\begin{OriginalVerbatim}[commandchars=\\\{\}]
\textcolor{nbsphinxin}{In [20]: }df.sort\PYGZus{}index(axis=1, ascending=False)
\end{OriginalVerbatim}

\begin{OriginalVerbatim}[commandchars=\\\{\}]
\textcolor{nbsphinxin}{In [21]: }df.sort\PYGZus{}values(by=\PYGZsq{}B\PYGZsq{})
\end{OriginalVerbatim}


\subsection{6.3 Selection}
\label{pandas_quickstart:6.3-Selection}
Getting

\begin{OriginalVerbatim}[commandchars=\\\{\}]
\textcolor{nbsphinxin}{In [22]: }df[\PYGZsq{}A\PYGZsq{}]
\end{OriginalVerbatim}

\begin{OriginalVerbatim}[commandchars=\\\{\}]
\textcolor{nbsphinxout}{Out[22]: }2013\PYGZhy{}01\PYGZhy{}01   \PYGZhy{}1.334013
         2013\PYGZhy{}01\PYGZhy{}02   \PYGZhy{}0.139974
         2013\PYGZhy{}01\PYGZhy{}03   \PYGZhy{}1.082739
         2013\PYGZhy{}01\PYGZhy{}04    0.034788
         2013\PYGZhy{}01\PYGZhy{}05    0.428547
         2013\PYGZhy{}01\PYGZhy{}06    0.125631
         Freq: D, Name: A, dtype: float64
\end{OriginalVerbatim}

\begin{OriginalVerbatim}[commandchars=\\\{\}]
\textcolor{nbsphinxin}{In [23]: }df[0:3]
\end{OriginalVerbatim}

\begin{OriginalVerbatim}[commandchars=\\\{\}]
\textcolor{nbsphinxin}{In [24]: }df[\PYGZsq{}20130102\PYGZsq{}:\PYGZsq{}20130104\PYGZsq{}]
\end{OriginalVerbatim}


\subsubsection{6.3.2 Selection by Label}
\label{pandas_quickstart:6.3.2-Selection-by-Label}
For getting a cross section using a label

\begin{OriginalVerbatim}[commandchars=\\\{\}]
\textcolor{nbsphinxin}{In [25]: }df.loc[dates[0]]
\end{OriginalVerbatim}

\begin{OriginalVerbatim}[commandchars=\\\{\}]
\textcolor{nbsphinxout}{Out[25]: }A   \PYGZhy{}1.334013
         B   \PYGZhy{}0.348297
         C    0.388654
         D   \PYGZhy{}0.225967
         Name: 2013\PYGZhy{}01\PYGZhy{}01 00:00:00, dtype: float64
\end{OriginalVerbatim}

Selecting on a multi-axis by label

\begin{OriginalVerbatim}[commandchars=\\\{\}]
\textcolor{nbsphinxin}{In [26]: }df.loc[:,[\PYGZsq{}A\PYGZsq{},\PYGZsq{}B\PYGZsq{}]]
\end{OriginalVerbatim}

Showing label slicing, both endpoints are included

\begin{OriginalVerbatim}[commandchars=\\\{\}]
\textcolor{nbsphinxin}{In [27]: }df.loc[\PYGZsq{}20130102\PYGZsq{}:\PYGZsq{}20130104\PYGZsq{},[\PYGZsq{}A\PYGZsq{},\PYGZsq{}B\PYGZsq{}]]
\end{OriginalVerbatim}

Reduction in the dimensions of the returned object

\begin{OriginalVerbatim}[commandchars=\\\{\}]
\textcolor{nbsphinxin}{In [30]: }df.loc[\PYGZsq{}20130102\PYGZsq{},[\PYGZsq{}A\PYGZsq{},\PYGZsq{}B\PYGZsq{}]]
\end{OriginalVerbatim}

\begin{OriginalVerbatim}[commandchars=\\\{\}]
\textcolor{nbsphinxout}{Out[30]: }A   \PYGZhy{}0.139974
         B   \PYGZhy{}1.347789
         Name: 2013\PYGZhy{}01\PYGZhy{}02 00:00:00, dtype: float64
\end{OriginalVerbatim}

For getting a scalar value

\begin{OriginalVerbatim}[commandchars=\\\{\}]
\textcolor{nbsphinxin}{In [31]: }df.loc[dates[0],\PYGZsq{}A\PYGZsq{}]
\end{OriginalVerbatim}

\begin{OriginalVerbatim}[commandchars=\\\{\}]
\textcolor{nbsphinxout}{Out[31]: }\PYGZhy{}1.3340127475498547
\end{OriginalVerbatim}

For getting fast access to a scalar (equiv to the prior method)

\begin{OriginalVerbatim}[commandchars=\\\{\}]
\textcolor{nbsphinxin}{In [32]: }df.at[dates[0],\PYGZsq{}A\PYGZsq{}]
\end{OriginalVerbatim}

\begin{OriginalVerbatim}[commandchars=\\\{\}]
\textcolor{nbsphinxout}{Out[32]: }\PYGZhy{}1.3340127475498547
\end{OriginalVerbatim}


\subsubsection{6.3.3 Selection by Position}
\label{pandas_quickstart:6.3.3-Selection-by-Position}
See more in Selection by Position Select via the position of the passed
integers

\begin{OriginalVerbatim}[commandchars=\\\{\}]
\textcolor{nbsphinxin}{In [33]: }df.iloc[3]
\end{OriginalVerbatim}

\begin{OriginalVerbatim}[commandchars=\\\{\}]
\textcolor{nbsphinxout}{Out[33]: }A    0.034788
         B   \PYGZhy{}0.677221
         C    0.120449
         D    0.794341
         Name: 2013\PYGZhy{}01\PYGZhy{}04 00:00:00, dtype: float64
\end{OriginalVerbatim}

By integer slices, acting similar to numpy/python

\begin{OriginalVerbatim}[commandchars=\\\{\}]
\textcolor{nbsphinxin}{In [34]: }df.iloc[3:5,0:2]
\end{OriginalVerbatim}

By lists of integer position locations, similar to the numpy/python
style

\begin{OriginalVerbatim}[commandchars=\\\{\}]
\textcolor{nbsphinxin}{In [35]: }df.iloc[[1,2,4],[0,2]]
\end{OriginalVerbatim}

For slicing rows explicitly

\begin{OriginalVerbatim}[commandchars=\\\{\}]
\textcolor{nbsphinxin}{In [36]: }df.iloc[1:3,:]
\end{OriginalVerbatim}

For slicing columns explicitly

\begin{OriginalVerbatim}[commandchars=\\\{\}]
\textcolor{nbsphinxin}{In [37]: }df.iloc[:,1:3]
\end{OriginalVerbatim}

For getting a value explicitly

\begin{OriginalVerbatim}[commandchars=\\\{\}]
\textcolor{nbsphinxin}{In [39]: }df.iloc[1,1]
\end{OriginalVerbatim}

\begin{OriginalVerbatim}[commandchars=\\\{\}]
\textcolor{nbsphinxout}{Out[39]: }\PYGZhy{}1.3477885295869219
\end{OriginalVerbatim}

For getting fast access to a scalar (equiv to the prior method)

\begin{OriginalVerbatim}[commandchars=\\\{\}]
\textcolor{nbsphinxin}{In [40]: }df.iat[1,1]
\end{OriginalVerbatim}

\begin{OriginalVerbatim}[commandchars=\\\{\}]
\textcolor{nbsphinxout}{Out[40]: }\PYGZhy{}1.3477885295869219
\end{OriginalVerbatim}


\subsubsection{6.3.4 Boolean Indexing}
\label{pandas_quickstart:6.3.4-Boolean-Indexing}
Using a single column’s values to select data.

\begin{OriginalVerbatim}[commandchars=\\\{\}]
\textcolor{nbsphinxin}{In [41]: }df[df.A \PYGZgt{} 0]
\end{OriginalVerbatim}

A where operation for getting.

\begin{OriginalVerbatim}[commandchars=\\\{\}]
\textcolor{nbsphinxin}{In [42]: }df[df \PYGZgt{} 0]
\end{OriginalVerbatim}


\paragraph{Using the isin() method for filtering:}
\label{pandas_quickstart:Using-the-isin()-method-for-filtering:}
\begin{OriginalVerbatim}[commandchars=\\\{\}]
\textcolor{nbsphinxin}{In [43]: }df2 = df.copy()
\end{OriginalVerbatim}

添加一列。

\begin{OriginalVerbatim}[commandchars=\\\{\}]
\textcolor{nbsphinxin}{In [44]: }df2[\PYGZsq{}E\PYGZsq{}] = [\PYGZsq{}one\PYGZsq{}, \PYGZsq{}one\PYGZsq{},\PYGZsq{}two\PYGZsq{},\PYGZsq{}three\PYGZsq{},\PYGZsq{}four\PYGZsq{},\PYGZsq{}three\PYGZsq{}]
\end{OriginalVerbatim}

\begin{OriginalVerbatim}[commandchars=\\\{\}]
\textcolor{nbsphinxin}{In [45]: }df2
\end{OriginalVerbatim}

\begin{OriginalVerbatim}[commandchars=\\\{\}]
\textcolor{nbsphinxin}{In [46]: }df2[df2[\PYGZsq{}E\PYGZsq{}].isin([\PYGZsq{}two\PYGZsq{},\PYGZsq{}four\PYGZsq{}])]
\end{OriginalVerbatim}


\subsubsection{6.3.5 Setting}
\label{pandas_quickstart:6.3.5-Setting}
Setting a new column automatically aligns the data by the indexes

\begin{OriginalVerbatim}[commandchars=\\\{\}]
\textcolor{nbsphinxin}{In [48]: }s1 = pd.Series([1,2,3,4,5,6], index=pd.date\PYGZus{}range(\PYGZsq{}20130102\PYGZsq{}, periods=6))
         s1
\end{OriginalVerbatim}

\begin{OriginalVerbatim}[commandchars=\\\{\}]
\textcolor{nbsphinxout}{Out[48]: }2013\PYGZhy{}01\PYGZhy{}02    1
         2013\PYGZhy{}01\PYGZhy{}03    2
         2013\PYGZhy{}01\PYGZhy{}04    3
         2013\PYGZhy{}01\PYGZhy{}05    4
         2013\PYGZhy{}01\PYGZhy{}06    5
         2013\PYGZhy{}01\PYGZhy{}07    6
         Freq: D, dtype: int64
\end{OriginalVerbatim}

Setting values by position

\begin{OriginalVerbatim}[commandchars=\\\{\}]
\textcolor{nbsphinxin}{In [49]: }df.iat[0,1] = 0
\end{OriginalVerbatim}

Setting by assigning with a numpy array

\begin{OriginalVerbatim}[commandchars=\\\{\}]
\textcolor{nbsphinxin}{In [50]: }df.loc[:,\PYGZsq{}D\PYGZsq{}] = np.array([5] * len(df))
\end{OriginalVerbatim}

The result of the prior setting operations

\begin{OriginalVerbatim}[commandchars=\\\{\}]
\textcolor{nbsphinxin}{In [51]: }df
\end{OriginalVerbatim}

A where operation with setting.

\begin{OriginalVerbatim}[commandchars=\\\{\}]
\textcolor{nbsphinxin}{In [52]: }df2 = df.copy()
\end{OriginalVerbatim}

\begin{OriginalVerbatim}[commandchars=\\\{\}]
\textcolor{nbsphinxin}{In [53]: }df2[df2 \PYGZgt{} 0] = \PYGZhy{}df2
\end{OriginalVerbatim}

\begin{OriginalVerbatim}[commandchars=\\\{\}]
\textcolor{nbsphinxin}{In [54]: }df2
\end{OriginalVerbatim}

\begin{OriginalVerbatim}[commandchars=\\\{\}]
\textcolor{nbsphinxin}{In [ ]: }
\end{OriginalVerbatim}
\begin{itemize}
\item {} 
Datasource-OpenStreetMap-OSM/TM/SRTM

\end{itemize}


\chapter{3、入门教程}
\label{index:id3}
GDAL的Geometry使用:

\url{http://nbviewer.jupyter.org/github/supergis/git\_notebook/blob/master/gdal/gdal-geometry.ipynb}


\chapter{高级数据分析}
\label{index:id4}

\chapter{Indices and tables}
\label{index:indices-and-tables}\begin{itemize}
\item {} 
\DUrole{xref,std,std-ref}{genindex}

\item {} 
\DUrole{xref,std,std-ref}{modindex}

\item {} 
\DUrole{xref,std,std-ref}{search}

\end{itemize}



\renewcommand{\indexname}{索引}
\printindex
\end{document}
